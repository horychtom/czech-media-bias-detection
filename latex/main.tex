% arara: pdflatex: { synctex: yes }
% arara: makeindex: { style: ctuthesis }
%% arara: bibtex

%\listfiles


%\PassOptionsToPackage{cp1250}{inputenc}

% The class takes all the key=value arguments that \ctusetup does,
% and couple more: draft and oneside
\documentclass[twoside]{ctuthesis}

% MY IMPORTS
\usepackage{pdfpages}

\makeatletter
\edef\mytoday{\expandafter\@gobbletwo\the\year\ifnum\month<10 0\fi\the\month\ifnum\day<10 0\fi\the\day}
\makeatother

% LaTeX logo with better kerning in sf bf font
\makeatletter
\newcommand\LaTeX@lmss@bx{L\kern -.33em{\sbox \z@ T\vbox to\ht \z@ {\hbox {\check@mathfonts \fontsize \sf@size \z@ \math@fontsfalse \selectfont A}\vss }}\kern -.15em\TeX}
\DeclareRobustCommand\myLaTeX{%
	\ifcsname LaTeX@\f@family @\f@series\endcsname
		\csname LaTeX@\f@family @\f@series\endcsname
	\else
		\LaTeX
	\fi
}


\ctusetup{
	mainlanguage = english,
	secondlanguage = czech,
	otherlanguages = {czech},
%   title-czech = {Bias Detection in Czech News},
	title-english = {Bias Detection in Czech News},
	doctype = B,
	faculty = F3,
	department-czech = {Katedra počítačů},
	department-english = {Department of Computer Science},
	author = {Tomáš Horych},
	supervisor = {Ing. Jan Drchal, Ph.D},
	fieldofstudy-english = {Open Informatics},
	subfieldofstudy-english = {Artificial Intelligence and Computer Science},
	fieldofstudy-czech = {Matematcké inženýrství},
	subfieldofstudy-czech = {Matematické modelování},
	keywords-czech = {zpracování přirozeného jazyka, detekce zaujatosti, mediální bias, detekce subjektivity, klasifikace textu},
	keywords-english = {natural language processing, bias detection, media bias,subjectivity detection, text classification},
	day = 10,
	month = 2,
	year = 2022,
	specification-file = {my_modules/assignment.pdf},
	front-specification = true,
    pkg-listings = true,
    ctulstbg = none,
}

\ctuprocess

% Theorem declarations, this is the reasonable default, anybody can do what they wish.
% If you prefer theorems in italics rather than slanted, use \theoremstyle{plainit}
\theoremstyle{plain}
\newtheorem{theorem}{Theorem}[chapter]
\newtheorem{corollary}[theorem]{Corollary}
\newtheorem{lemma}[theorem]{Lemma}
\newtheorem{proposition}[theorem]{Proposition}

\theoremstyle{definition}
\newtheorem{definition}[theorem]{Definition}
\newtheorem{example}[theorem]{Example}
\newtheorem{conjecture}[theorem]{Conjecture}

\theoremstyle{note}
\newtheorem*{remark*}{Remark}
\newtheorem{remark}[theorem]{Remark}

% Marginpars used as navigation aids.
\usepackage{mparhack}
\usepackage{hyperref}
\usepackage[autostyle]{csquotes} 
\usepackage{subfig}
\usepackage{svg}
\usepackage{spverbatim}
\usepackage{fancyvrb}
\usepackage{fvextra}

% glossary input
\usepackage{glossaries}

\makeglossaries
\newacronym{nup}{NÚP}{\textit{Nezaujatý Úhel Pohledu}}
\newacronym{mb}{MB}{Media Bias}
\newacronym{iaa}{IAA}{Inter-Annotator Agreement}
\newacronym{mbg}{MBG}{Media Bias Group}
\newacronym{npov}{NPOV}{Neutral~Point~Of~View}
\newacronym{wnc}{WNC}{Wiki Neutrality Corpus}
\newacronym{nfnz}{NFNZ}{\textbf{Nadační fond nezávislé žurnalistiky}}
\newacronym{nmt}{NMT}{Neural Machine Translation}
\newacronym{nlp}{NLP}{Natural Language Processing}
\newacronym{ml}{ML}{Machine Learning}
\newacronym{mtl}{MTL}{Multi-Task Learning}
\newacronym{nn}{NN}{Neural Network}
\newacronym{ann}{ANN}{Artificial Neural Network}
\newacronym{mlp}{MLP}{Multi-Layer Perceptron}
\newacronym{rnn}{RNN}{Recurrent Neural Network}
\newacronym{rnns}{RNNs}{Recurrent Neural Networks}
\newacronym{sota}{SOTA}{State-Of-The-Art}
\newacronym{mlm}{MLM}{Masked Language Modelling}
\newacronym{nsp}{NSP}{Next Sentence Prediction}
\newacronym{cv}{CV}{cross-validation}
\usepackage{colortbl}
\definecolor{unbiased_clr}{RGB}{183,235,198}
\definecolor{biased_clr}{RGB}{236,189,189}
\definecolor{biasedw_clr}{RGB}{225,130,130}

\definecolor{highlight}{RGB}{198,220,240}


\newcommand\indexmp[1]{{\sffamily\bfseries#1}}

\ExplSyntaxOn
\cs_new:Nn \ctuman_domarginpar:n {
	\marginpar
	[ \raggedleft \footnotesize \sffamily #1 ]
	{ \raggedright \footnotesize \sffamily #1 }
}
\cs_generate_variant:Nn \ctuman_domarginpar:n { x }
\DeclareDocumentCommand \ctump { m } {
	\clist_set:Nn \ctuman_temp_clist { #1 }
	\ctuman_domarginpar:x { \clist_use:Nnnn \ctuman_temp_clist { \\ } { \\ } { \\ } }
	\clist_map_inline:Nn \ctuman_temp_clist { \index{##1|indexmp} }
	\ignorespaces
}
\ExplSyntaxOff



% Abstract in Czech
\begin{abstract-czech}
Automatická detekce zaujatosti v médiích představuje možnou cestu k objektivnějšímu a faktičtějšímu psaní. Tato práce se zaměřuje na zaujatost médií a zabývá se problémem binární klasifikace zaujatosti médií v českém zpravodajském prostředí. Je zde proveden přehled literatury a metodiky pro různé aspekty zaujatosti. Následně je shromážděn a analyzován soubor dat týkajících se zajuatosti médií. S využitím strojového překladu je prezentováno osm paralelních českých datasetů, přičemž jeden z nich je rozsáhlý dataset o 360 tisících větách. Kromě toho je z wikipedie automaticky extrahován nový český dataset CWNC (Czech Wiki Neutrality Corpus) pro detekci zaujatosti, s 5 766 větami. Následné experimenty pak ukazují vliv předtrénování na kombinacích aktuálních dostupných doménových datasetů a naznačují pozitivní vliv předtrénování na datasetech zaměřených na subjektivitu. Český BERT model s nejlepšími parametry a nastavením pak dosahuje skóre F1 \textbf{80,2\%} na vybraném testovacím dataset. Nakonec je natrénovaný klasifikátor využit ke klasifikaci reálných dat z různých zpravodajských zdrojů.

\end{abstract-czech}

% Abstract in English
\begin{abstract-english}
Automatic detection of bias in media represents a possible way to more objective and factual writing. This work focuses on \textbf{media bias} and addresses the problem of binary classification of media bias in the Czech news environment. The literature and methodology for different aspects of bias are reviewed. Then, a set of datasets related to media bias is collected and analyzed. Utilizing machine translation, eight parallel Czech datasets are presented, one of which being a large-scale dataset of 360k sentences. Additionally, a novel Czech dataset CWNC (Czech Wiki Neutrality Corpus) for bias detection with 5,766 sentences is automatically extracted from Wikipedia. Following experiments then show the effects of pre-training on combinations of current in-domain datasets, suggesting a positive effect of pre-training on subjective datasets. Czech BERT-based model with the best parameters and setting then achieves an F1 score of \textbf{80.2\%} on a selected target dataset. Finally, a trained classifier is utilized to classify real-world data from various news sources.

\end{abstract-english}

% Acknowledgements / Podekovani
\begin{thanks}
I thank to my supervisor, Jan Drchal, for the guidance through the Machine Learning and NLP methodology and for supporting my research interests. I would also like to thank to my family, who always supported me during my studies and throughout the process of work on this thesis.
\end{thanks}

% Declaration / Prohlaseni
\begin{declaration}
I declare that the presented work
was developed independently and that
I have listed all sources of information
used within it in accordance with the
methodical instructions for observing
the ethical principles in the preparation
of university theses.

\medskip

In Prague, \monthinlanguage{title} \ctufield{day}, \ctufield{year}

\medskip

\end{declaration}

\usepackage{url}
\usepackage{subfig}

\usepackage{tabularx,array}
\usepackage{mathtools,amssymb}
\usepackage{amsmath,amsthm,enumitem}
\usepackage{comment}


% A savebox for typesetting listings in the titles
\newsavebox{\myboxa}

%\newcommand*\symbO{$\color{red}\bowtie$}
\newcommand*\symbO{\raisebox{0.5\height}{\scalebox{0.7}{\color{red}${\vartriangleright}\mkern-6mu{\vartriangleleft}$}}}
\newcommand*\symbM{\raisebox{0.5\height}{\scalebox{0.7}{\color{red}${\blacktriangleright}\mkern-6mu{\blacktriangleleft}$}}}
\newcommand*\itemO{\item\leavevmode\kern-0.33em\symbO}
\newcommand*\itemM{\item\leavevmode\kern-0.33em\symbM}


\begin{document}

% We actually don't want inline listings to have a background color
\renewcommand \ctulstsep {0pt}

% \ctuclsname for typesetting the class' name
\newcommand\ctuclsname{\leavevmode\unhcopy\ctuclsnamebox}
\newsavebox\ctuclsnamebox
\begin{lrbox}{\ctuclsnamebox}
\ctulst!ctuthesis!
\end{lrbox}

% all the formal stuff (abstract, title page etc)

\maketitle















%-------------------------------------------------------------------------------------------------------------------
% MAIN TEXT
\chapter{Introduction}
This is introduction to my thesis, motivation. taky něco o nlp
\section{Motivation}
tady něco o nlp
\section{My contribution}
\section{Outline}



\chapter{Bias Detection}
Before turning my attention to media bias, I have examined several other relevant bias detection topics. At the beginning of my research, I studied the possibilities of applying gender bias detection to Czech News. Therefore, the following small section is dedicated to my results and examination of one of the gender-focused datasets.


%______________________________GENDER______________________________________________%
\section{Gender bias detection}\label{gender}
Most of the work on gender bias aims to study gender bias embedded in models and other methods to measure, clarify, and possibly mitigate it.

There is clear evidence that current language models possess implicit gender bias. Whether it means, in terms of learning biased embeddings \cite{bolukbasi2016man}, or simply underrepresentation of a particular gender in the data \cite{sun-peng-2021-men}. 

Yet, my work aspires to classify news texts; therefore, I examined the possibilities of gender classification in text.

I closely followed the approach of Dinan et al. \cite{dinan2020multi}. They define three dimensions of gender bias: bias when speaking \textit{ABOUT} someone, \textit{TO} someone or \textit{AS} someone. Target classes are \{masculine,feminine,neutral\}. 

The \textit{bias} here simply means an aspect of the statement that implies the gender of a particular person along the dimensions. To make this definition more clear, for example, the authors further propose that an unbiased sentence would be a sentence in which a machine learning model would not be able to classify a gender because there would basically be no difference between the classes. Yet, in a real-world scenario, sentences \textbf{are} influenced by gender, and therefore such classification is possible.

To measure this kind of bias over all three dimensions, a large-scale dataset \textbf{md\_gender}\footnote{\url{https://huggingface.co/datasets/md_gender_bias}} has been collected. The authors train a transformer model using \Gls{mtl}, to capture all three dimensions. However, only the \textit{ABOUT} dimension and a very small fraction of the \textit{AS} dimension are publicly available, so I focused only on the \textit{ABOUT} dimension.

\newpage

\begin{itemize}
\item \textbf{md\_gender} - is a collection of automatically labeled large-scale data gathered from various sources around the internet, where gender annotation of a particular dimension is provided (eg., gender information of a user in an internet discussion). It also includes a small gold-labeled dataset for evaluation with 785 data points for the \textit{ABOUT} dimension.
\end{itemize}

\subsection{Initial experiment}
To transfer the results of the paper mentioned above to the Czech environment, I sampled 150k sentences from across all datasets with an \textit{ABOUT} dimension label and translated them via \textbf{DeepL} machine translator (more on machine translation in section \ref{DeepL}). Then I managed to train a RoBERTa-based model \cite{liu2019roberta} that achieved an F1 score of 80\% on the small gold-labeled evaluation dataset. 
Unfortunately, the results are not comparable to the original English experiments because I took a \textbf{single-task} approach and omitted other dimensions completely. I share the trained model together with translated data on HuggingFace\footnote{\url{https://huggingface.co/}} hub, and I also present a demo. An example of the demo can be seen in Appendix \ref{gender_results}.


\subsection{Discussion}
The gender classifier, such as this one, can be used to determine what percentage of a particular article in the Czech news environment is about men, women, or is completely genderless. This statistical indicator could help to keep the writing more balanced or provide insight into already published writings.

Moreover, gender bias could also be an interesting task choice in the \gls{mtl} setting, with respect to media bias. However, since this is just an initial experiment and is not further developed, I suggest that a clear-cut methodology and datasets review should be performed. 



%___________________________MEDIA_BIAS______________________________________________%
\section{Media bias detection}\label{mediabias}
When it comes to automatic detection of media bias, the standard is to use supervised learning. Most of the previous work in media bias used hand-crafted features together with traditional\footnote{By traditional I refer to all \Gls{ml} models that are not deep neural networks.} \Gls{ml} algorithms. For example, Hube et al. \cite{hube2018detecting} used a lexicon-based approach with various lexicons (sentiment, bias, subjective, and other linguistic features). Although hand-crafted feature-based approaches offer fairly reasonable explainability of the model's decision, they were outperformed by neural networks and have been replaced by them completely.

The majority of current research focuses on \textbf{sentence-level} classification \cite{sinha2021determining,Spinde2021MBIC,lee2021unifying,hube2019neural}, however, there has also been an effort to lift the classification to the \textbf{article-level}.

Article-level classification is usually more difficult since it is inconvenient to put the whole article through the neural network. Even though such things as document embeddings exist \cite{lau2016empirical}, bottom-up solutions are usually used. A simple approach would be to classify all sentences and count the frequency. Eventually, I used this approach when applying the classifier to Czech news corpora in section \ref{inference}.

However, additional high-level features such as the position of bias, frequency, or ordering, have been studied and proved to be effective in article-level classification \cite{chen2020detecting,chen-etal-2020-analyzing}.

As I outlined in the previous section, \gls{mb} can be divided into two classes, where one depends on the outer context, and the other does not. This is commonly referred to as \textbf{informational} and \textbf{lexical} bias. There have been efforts to classify informational bias with varying context sizes \cite{van2020context}, although a majority of the work focuses on lexical bias, and I follow this standard as well.

Various pre-training and fine-tuning strategies have been studied regarding sentence classification. However, one of the most promising approaches is using an \Gls{mtl} to tackle the problem. Although there are already some results of applying \Gls{mtl} to the detection of \gls{mb} \cite{lee2021unifying,spindeexploiting}, empirical studies suggest that a large number of tasks should be used to allow \Gls{mtl} to shine \cite{aribandi2021ext5}. For more details about \gls{mtl} see \ref{theory} section.

\let\cleardoublepage\clearpage

\chapter{Datasets} \label{datasets}
Due to the complex nature of \gls{mb}, different datasets try to capture different aspects of it. In this section, I present a collection of \textbf{all} datasets related to biased writing and subjectivity detection available. Later this collection is used to study the proposed research questions \hyperref[Q1]{Q1} and \hyperref[Q2]{Q2} and to build a final classifier. For details see experiment section \ref{experiments}.

As stated above, this work focuses only on sentence-level classification; thus, data annotated on article-level were not considered.

I divided the available datasets into 3 main families: 
\begin{itemize}
    \item Subjectivity bias
    \item Wikipedia bias
    \item Media bias
\end{itemize}
Wikipedia bias could also be considered as a form of subjective bias, but all the Wiki data come from the same distribution\footnote{some are even different samples from the same larger corpora}. and environment; hence I find it reasonable to put them together.


\section{Subjectivity Datasets}

\subsection{SUBJ}
It is reasonable to include datasets that focus on the detection of subjectivity since it is one of the \gls{mb} characteristics. The Subjectivity dataset (SUBJ) \cite{Pang+Lee:04a} consists of 10000 sentences gathered from movie review sites. Sentences are labeled as subjective and objective with 1:1 ratio. 

The data were collected in an automatic way. The authors made an assumption that all reviews from Rottentomatoes\footnote{https://www.rottentomatoes.com/} are subjective, and all plot summaries from IMBD\footnote{ www.imdb.com} are objective. For each class, 5k sentences were sampled \textbf{randomly}.




\subsection{MPQA}
\textbf{M}ulti-\textbf{P}erspective \textbf{Q}uestion \textbf{A}nswering (MPQA) Opinion corpus is another dataset that can be used for subjectivity detection. I used the MPQA Opinion corpus version 2.0, which consists of 692 articles from 187 different news sources. In total 15802 sentences. All articles are from June 2001 to May 2002.

The corpus offers a rich annotation scheme \cite{wiebe2005annotating} that focuses on sentiment and subjectivity annotations.

To extract the bias information, I focused on two types of annotations:
\begin{itemize}
    \item Direct subjective
    \item Expressive subjective
\end{itemize}
These annotations were present if any form of subjectivity was suspected by the annotator. Each annotation consists of indices of span in the text and properties. For each sentence in the corpus, I extracted labels as follows:

If there was at least one annotation \textbf{direct\_subjective} or \textbf{expressive\_subjectivity} with span inside the sentence and the intensity tag was not $low$, the sentence was labelled as \textit{subjective $\sim$ biased}. All other sentences were extracted as \textit{objective $\sim$ unbiased}.

This approach has produced $9484$ subjective sentences and 6318 objective sentences.


\section{Media Bias datasets}


\subsection{BASIL}
BASIL dataset \cite{fan2019plain} comprises 300 articles with 1727 sentence level bias annotations. The authors of the dataset distinguish between \textbf{lexical} and \textbf{informational} bias.

The annotations were performed by two experts and further resolution discussions have later led to 0.56 and 0.7 \Gls{iaa} score for lexical and informational bias, respectively.

Even though BASIL brings sufficient annotation quality, most of the labeling resulted in informational bias annotations, leaving only 478 sentences for the lexical bias class. Informational bias requires a different approach to detection \cite{van2020context} and usually depends dramatically on the context. Therefore, I extracted all sentences with the informational label as a neutral class.




\subsection{Ukraine Crisis Dataset}
This dataset \cite{farber2020multidimensional} offers 2057 sentences with binary media bias labels. All sentences are related to one topic - the Ukraine-Russian crisis. Data were gathered from 90 news sources.

The authors introduce rich annotations for each sentence. Each of them looks at the bias from a different perspective, called \textit{bias dimensions}:
\begin{enumerate}
    \item Hidden Assumptions and Premises
    \item Subjectivity
    \item Framing
\end{enumerate}
In addition, the \textit{overall bias} annotation is presented. Together, the data include 44547 fine-grained annotations. For simplicity, I only included the overall bias annotation.
Even though this dataset encompasses comprehensive bias information, it also suffers from a low \Gls{iaa} score. Specifically, Krippendorff’s $\alpha = -0.05$.



\subsection{NFNJ}
The NFNJ\footnote{\cite{farber2020multidimensional} refers to this dataset as NFNJ, however, in the original paper the name is not presented.} dataset provides 966 sentences from 46 articles with annotations on a fine-grained level.

Authors share the dataset for research purposes; however, the public version differs from the one described in the original paper. Therefore, while extracting the final dataset, I made a few assumptions:

In the raw data, contributions from multiple annotators on each sentence are provided. Therefore, I extracted the labels as a simple arithmetical mean of the labels. Furthermore, the original labels stand for 
\begin{itemize}
    \item 1: 'neutral'
    \item 2: 'slightly biased but acceptable'
    \item 3: 'biased'
    \item 4: 'very biased'
\end{itemize}
To obtain the final labels in an unbiased-biased format, I simply assumed sentences with mean-score $\leq$ 2 as neutral and $>$ 2 as biased.

The Fleiss Kappa \Gls{iaa} score averaged at zero, making it practically unusable as a standalone dataset.



\subsection{BABE}
\textbf{B}ias \textbf{A}nnotations \textbf{B}y \textbf{E}xperts (BABE)\cite{spinde2021neural} is a key media bias dataset from \Gls{mbg}\footnote{\url{https://media-bias-research.org/}}, which is to the best of my knowledge, the highest quality media bias dataset to this day. It builds on top of MBIC \cite{Spinde2021MBIC} which is a smaller crowd-sourced dataset.

BABE contains 3700 sentences. 1700 sentences are from MBIC, which were extracted from 1000 news articles, and in addition extended by 2000 more sentences, altogether covering 12 topics, annotated with binary bias indications. In addition, the annotations were enriched with a list of biased words. However, the presence of biased words does not always result in an overall biased sentence label. See \ref{table:2} for examples.

It has been annotated by eight experts resulting in \gls{iaa} Krippendorfs $\alpha = 0.39$, which exceeds other media bias datasets by a significant margin. It also provides detailed information about the annotator background, making it a \textbf{reliable} source of information. The pipeline of the collection of BABE can be seen in \ref{fig:babe-data}.

This dataset plays a pivotal role in my approach to media bias detection and is selected as a target for tuning and evaluating language models in chapter \ref{experiments}. Examples of BABE data points can be seen in 
\ref{table:2}

\begin{table}
\makebox[\textwidth][c]{
\begin{tabular}[\textwidth]{l|c}
\hline
\textbf{sentence} & \textbf{label} \\
\hline
\cellcolor{biased_clr} Americans \colorbox{biasedw_clr}{know} President Donald Trump is an \colorbox{biasedw_clr}{outrageous}, \colorbox{biasedw_clr}{scandal-ridden} character. & \cellcolor{biased_clr} biased \\
\hline
\cellcolor{unbiased_clr} Biden said he would seek Muslims to serve in his administration. & \cellcolor{unbiased_clr} unbiased \\
\hline
\cellcolor{biased_clr} Biden’s shift \colorbox{biasedw_clr}{radically} leftward reflects that of his party. & \cellcolor{biased_clr} biased \\
\hline
\cellcolor{unbiased_clr} Anti-vaccine groups take \colorbox{biasedw_clr}{dangerous} online \colorbox{biasedw_clr}{harassment} into the real world. & \cellcolor{unbiased_clr} unbiased \\
\hline
\end{tabular}
}
\caption{Example of biased and unbiased sentences from \textbf{BABE}}
\label{table:2}
\end{table}


\begin{figure}
  \includegraphics[scale=0.3]{my_modules/multimedia/babe_workflow.png}
  \caption{Data collection and annotation pipeline of \textbf{BABE}, reprinted from \cite{Spinde2021f}}
  \label{fig:babe-data}
\end{figure}



\section{Wikipedia datasets}\label{wiki-npov}
Due to annotation costs and the overall lack of large-scale datasets in the media bias setting, many researches \cite{pryzant2020automatically,recasens2013linguistic,hube2019neural} used Wikipedia's \Gls{npov} policy\footnote{\url{https://en.wikipedia.org/wiki/Wikipedia:Neutral_point_of_view}} to construct a large-scale corpora \textbf{automatically}. 

Wikipedia's NPOV policy is a set of rules that aim to preserve neutrality in Wikipedia articles. Some examples of NPOV principles are as follows:
\begin{itemize}
    \item Avoid stating opinions as facts.
    \item Avoid stating facts as opinions.
    \item Prefer nonjudgmental language.
\end{itemize}

When neutrality is contested, a Wikipedia article can be moved to NPOV dispute by tagging it with \{\{NPOV\}\} or \{\{POV\}\}\footnote{Other POV related variations are often used.} template. Debate on specific details of neutrality violations is then initialized among editors and eventually resolved, leading to the removal of the tag.

This editorial information can be leveraged to extract parts of the text that violate the NPOV and their unbiased counterparts. However, it has been shown \cite{hube2019neural,zhong-etal-2021-wikibias-detecting} that such automatic extraction can suffer from noisy labeling. In some cases \cite{hube2019neural} up to 60\% of data points from the positive class were actually neutral.

Even though these datasets introduce a large number of samples that are highly related to media bias, they are all sampled from Wikipedia's environment, which can be very different from the news environment. For this reason, a dataset based \textbf{only} on Wikipedia should not be used for training a final classifier of the \textbf{news text}.




\subsection{Wiki Neutrality Corpus}\label{wiki}
\Gls{wnc} \cite{pryzant2020automatically} is a parallel corpus of 180k pairs of biased and unbiased sentences. For the collection of the data, \ref{wiki-npov} approach was adopted. The authors crawled revisions from 2014 to 2019. Each revision has been processed to check if it contains any variation of \textit{POV} related tag inside. This approach yielded 180k pairs such that the sentence before edit is considered biased and the modified/added sentence after edit is considered neutral/unbiased.
    
In addition to WNC, 385k of sentences that have not been changed during the NPOV dispute were extracted as neutral and for word-level classification purposes, a subset of the WNC corpus, where only one word is changed in the biased-unbiased pair, were added.




\subsection{CW-HARD}
Hube et al. \cite{hube2019neural} constructed a dataset based on NPOV, where only revisions with one sentence diff were filtered. However, because of the potentially noisy outcome, 5000 sentences were sampled and annotated using crowdsourcing. However, the Krippendorff's Alpha agreement score measured only $\alpha = 0.124$, which is generally considered low. 

After filtering out sentences that annotators labeled with the "I don't know" option, the final dataset consists of 1843 statements labeled as biased and 3109 labeled as neutral, a total of 4953 sentences.




\subsection{WikiBias}
This is the latest dataset based on Wikipedia. The authors \cite{zhong-etal-2021-wikibias-detecting} closely follow the approach of WNC \cite{pryzant2020automatically} and extract another parallel wiki corpus of 214k sentences.
To achieve a higher quality corpus, 4099 sentence pairs were randomly sampled and labeled by trained annotators. As a result, introduced \textbf{WikiBias-Manual} dataset consists of 3400 biased and 4798 neutral sentences annotated with high \gls{iaa} score of Cohen's $\kappa = 0.734$

\begin{table}
\begin{ctucolortab}
\begin{tabular}{c|c|c|c}
 \textbf{Dataset} & \textbf{Size} & \textbf{Annotation} & \textbf{Agreement}\\
 \hline
 \textbf{SUBJ} & 10.000 & automatic & -\\ 
 \hline
 \textbf{MPQA} & 15.802 & annotators & high \\
 \hline
 \textbf{BASIL} &  1.727 & annotators & medium \\ 
 \hline
 \textbf{Ukraine Crisis Dataset} & 2.057 & crowdsourcing & low \\ 
 \hline
 \textbf{NFNJ} & 888 & crowdsourcing & low \\
 \hline
 \textbf{BABE} & 3673 & annotators & medium \\
 \hline 
 \textbf{WNC} & 362.990 & automatic & - \\
 \hline
 \textbf{CW-hard} & 4953 & crowdsourcing & low \\
 \hline 
 \textbf{WikiBias} & 8198 & annotators & high \\
 \hline
\end{tabular}
\end{ctucolortab}
\caption{Comparison of all bias related datasets collected}
\label{table:1}
\end{table}



\section{Unused datasets}
 Some datasets focus on a slightly different task yet still carry potentially useful information. Such data can be useful in a \gls{mtl} setting \ref{mtl}. To name a few which are focused on the detection of ideology:
\begin{itemize}
\item \textbf{NewsB} - 
Consists of labels capturing the authors political ideology (liberal, conservative) Labeled through distant supervision.
\item \textbf{IBC} - Also focuses on ideology detection; however, it is not publicly available.
\end{itemize}


\section{Summary}
In the previous section, I introduced all resources that are potentially useful for media bias analysis and are publicly available. The overview of all datasets and its properties can be seen in figure \ref{table:1}.

BABE dataset is generally a good benchmark, and its translated parallel version will be used for evaluation in this work. Training a classifier on combinations of different datasets is studied in \ref{experiments}. Unfortunately, a lot of the data suffer from noisy labeling and low \gls{iaa} greatly. Therefore, its usability is considerably limited.


\chapter{Czech datasets}
\section{Translation}
\subsection{DeepL}
\section{Processing}
\section{Analysis}
output \= unified set of czech bias related datasets
\textbf{Czech Unified set of Bias Data}
\begin{enumerate}
    \item mpqa-cs
    \item subj-cs
    \item newsb-cs
    \item cw-hard-cs
    \item wiki-npov-large-cs
    \item babe-cs
\end{enumerate}

\section{Czech-Wiki dataset }

\chapter{Theoretical background}
\section{Text classification}
\subsection{Metrics}
\begin{itemize}
    \item F1 \label{f1}
\end{itemize}
\section{Neural Networks }
\section{Attention and transformers}\label{att_transformers}
\section{Transfer learning}


\chapter{Experiments}\label{experiments}
Finally, to complete the last research task proposed in the introduction \hyperref[problem_definition]{T4}, the datasets collected in section \ref{datasets} are leveraged to build a \textbf{media bias classifier}.

To do so, a transformer model (\ref{att_transformers}) is \textbf{fine-tuned} on the BABE dataset (\ref{babe}).


Furthermore, to answer the two research questions,\hyperref[Q1]{Q1} and \hyperref[Q2]{Q2} and to push the performance of the classifier, the effect of further pre-training\footnote{Primary pre-training is the original unsupervised pre-training task executed by the authors of the particular language model.} are studied.

A scheme of the whole process can be seen in figure \ref{fig:pipeline}. Firstly, a baseline fine-tuning is performed to select the suitable language model. Then a limited hyperparameter tuning of the model follows. Subsequently, the tuned model is then pre-trained on combinations of auxiliary datasets reviewed in section \ref{datasets}. The impact of the pre-training is studied through 1) direct evaluation on the BABE and 2) fine-tuning and evaluation on the BABE.

The optimal pre-training strategy and hyperparameters are then used to build the final clasifier.
 
\begin{figure}[h]
  \includegraphics[scale=0.5]{my_modules/multimedia/pipeline.png}
  \caption{Scheme of experiments.}
  \label{fig:pipeline}
\end{figure}
 
\section{Models}
Architecture-wise, a classifier consisting of a dense\footnote{Sometimes is referred to as fully connected layer.} layer is attached to the pre-trained language model to perform the binary classification task. The tested language models were pre-trained either solely on Czech data (monolingual) or on multiple languages jointly (multilingual). The scheme of the particular architecture used can be seen in figure \ref{fig:classifier}.

As opposed to English language, there is a relatively low number of Czech pre-trained language models available. A list and a brief summary of all the models tested can be found in the following:



\begin{figure}
\makebox[\textwidth][c]{
  \includegraphics[scale=0.3]{my_modules/multimedia/final_colored.png}
  \caption{Scheme of a text classification architecture used in the fine-tuning a BERT-based models.}
  \label{fig:classifier}
  }
\end{figure}



\begin{itemize}
    \item \textbf{RobeCzech} \cite{strakarobeczech} - RoBERTa-based model with 125M parameters. Like its original counterpart, it is trained with the \gls{mlm} task, on 4,917M tokens of Czech corpora.
    \item \textbf{Czert} \cite{sido-etal-2021-czert} - BERT-based model with 110M parameters, trained with \gls{nsp} tasks. All Together trained on 37GB of Czech text. 
    \item \textbf{FERNET-C5} \cite{lehevcka2021comparison} - BERT-based model trained with the \gls{mlm} and \gls{nsp} task on 93GB of Czech text from the Common Crawl project.
    \item \textbf{FERNET-News} \cite{lehevcka2021comparison} - RoBERTa-based model trained with \gls{mlm} task on 20GB of Czech News text.
    \item \textbf{SlavicBert} \cite{arkhipov2019tuning} - BERT-based model with 179M parameters, trained on four languages: Russian, Bulgarian, Czech, and Polish. The model is trained on all 4 languages at once. The model is not trained from scratch, but it is a fine-tuned version of mBERT.
    \item \textbf{mBERT} \cite{devlin2019bert} - BERT-based model with 179M parameters trained on corpora of 104 languages, including Czech, with MLM task.
\end{itemize}







\section{Experimental setup}
All models are fetched, trained, and evaluated using the HuggingFace API\footnote{\url{https://huggingface.co/docs}}. The maximum sequence length is set to 128 tokens. All training parameters can be seen in the Appendix \ref{all_parameters}.

A small portion (15\%) of the target dataset is left aside as a \textbf{test set} at the beginning and is used only for the final evaluation to ensure that no test data leak into the training data.

Every fine-tuning, except the one performed on the final test set, is evaluated using a 10-fold \gls{cv}. This helps to get more realistic estimates of model performance than a simple train-validation split would give. The only evaluation metric used for all experiments is the F1 score with \textit{macro} averaging\footnote{The F1 score is computed for both classes and averaged.}. 

All training has been done on the RCI cluster node with 4 x NVIDIA Tesla V100 with 32GB GPU graphic memory.





 \section{Baseline}
 As a baseline, all Czech and multilingual models listed are fine-tuned on BABE and evaluated using a 10-fold stratified \gls{cv}. 
 
 Because of the novelty of the CWNC, I also perform a baseline evaluation over this dataset, but later it is only used as an auxiliary dataset.
 
 The hyperparameters used are the same as those used by the authors of BABE \cite{Spinde2021MBIC}. However, the authors used early stopping together with \gls{cv} and used the validation split inside \gls{cv} to early stop. This may lead to data leakage. Which can subsequently lead to too optimistic results.
 
 A solution to this problem would require another split for validation, but at this point, the size of the training data is already shrunk significantly. Therefore, I did not use early stopping together with \gls{cv} at all. Instead, I fixed the number of epochs to 3 as suggested by the authors of BERT \cite{devlin2019bert} . 
 All other hyperaparemeters remained unchanged; AdamW optimizer is used with an initial learning rate of 5e-5 and a batch size of 64.
 
 The baseline evaluation of all Czech models used can be seen in table \ref{table:3}. The final F1 score is averaged across all folds.
 
 The model that performs best on the BABE is \textbf{FERNET-C5}. It also performs best on the novel CWNC dataset; therefore, it is a suitable candidate for further tuning. From now on, all experiments are performed using this model.
 

 \begin{table}
\makebox[\textwidth][c]{
\begin{ctucolortab}
\begin{tabular}{c||c|c|c|c|c|c}
 \textbf{target}\textbackslash \textbf{models} & \textbf{Czert} & \textbf{RobeCzech} & \textbf{mBERT} & \textbf{FERNET-C5} & \textbf{FERNET-News} & \textbf{SlavicBERT}\\
 \hline
 \hline
 \textbf{BABE} & 0.776 & 0.774 & 0.733 & \textbf{0.781} & 0.566 & 0.754 \\
 \hline
 \textbf{CWNC} & 0.732 & 0.709 & 0.734 & \textbf{0.747} & 0.443 & 0.741 \\
 \hline
 \hline
 \textbf{mean} & 0.754 & 0.742 & 0.734 & \textbf{0.764} & 0.505 & 0.748 \\
\end{tabular}
\end{ctucolortab}
\caption{F1 scores of baseline fine-tuning. Best scores for each dataset are highlighted.}
\label{table:3}
}
\end{table}

 
 
 
 
 \section{Hyperparameter tuning}
I restricted the search space of hyperparameters only to the combinations of:
 \begin{itemize}
     \item \textbf{Batch size} $\in \{16,32\}$
     \item \textbf{Learning rate} $\in $ \{2e-5,3e-5,5e-5\}
     \item \textbf{Epochs} $\in \{2,3,4\}$
 \end{itemize}
 
 As the authors of the original BERT paper suggest \cite{devlin2019bert}. Then I ran a grid search with \gls{cv}. The overall best parameters were as follows:
 \begin{center}
      \{learning\_rate = 3e-5,batch\_size = 32, epochs=3\}\label{hyperparams}
 \end{center}
 
 The model with the best parameters achieved a 0.784 F1 score ($\sim$0.4\% improvement against baseline).


 
 





\section{Combining Datasets}
This section is dedicated to the study of the influence of pre-training on combinations of datasets. Trying all combinations would result in training 511 models\footnote{Given a set of $n$ elements, number of subsets is $2^n$. Here, we have a set of nine datasets, resulting in 512 subsets. 511 without an empty set.}, which is infeasible. Therefore, I decided to experiment with pre-training on five subsets of datasets with regard to their bias information, to see which of them can serve as a good initialization for fine-tuning on BABE.
The combination sets are as follows:
\begin{itemize}
    \item \textbf{SUBJe} - is a combination of the SUBJ and MPQA dataset, both of which focus on explicit subjective bias.
    \item \textbf{MB} - is a combination of NJNJ, UA-crisis and BASIL dataset which are all from the \gls{mb} family.
    \item \textbf{WIKI} - are all datasets collected from Wikipedia. It consists of CW-hard, WikiBias, and CWNC. The three datasets were collected automatically with respect to NPOV violations as described in \ref{wiki-npov}.
    \item \textbf{ALL} - This one is simply a combination of all datasets except the WNC.
    \item \textbf{WNC} - WNC is almost 90\% of all data; therefore, I perform experiments on this dataset separately.
\end{itemize}
 
In every combination, the data were randomly mixed and subsequently downsampled, so that the classes were balanced. For each combination, 20\% of data were used as a validation set to decide the optimal number of epochs for pre-training. The convergence of validation losses can be seen in the figure \ref{fig:all_losses}. The number of epochs for pretraining were chosen as follows: 1, 3, 1, 2, 1 for SUBJe, MB, WIKI, ALL and WNC respectively.

This procedure yields five pre-trained models for further experiments.

\begin{figure}
  \includegraphics[scale=0.5]{my_modules/multimedia/all_losses.png}
  \caption{Convergence of validation loss over different dataset combinations}
  \label{fig:all_losses}
\end{figure}

\begin{table}
\makebox[\textwidth][c]{
\begin{ctucolortab}
\begin{tabular}{c||c|c|c|c|c|c|}
  & \textbf{BABE} & \textbf{SUBJ} & \textbf{WIKI} & \textbf{MB} & \textbf{WNC} & \textbf{ALL}\\
 \hline
 \hline
 \textbf{Pre-trained + Fine-tuned} & 0.7835 & 0.7875 & 0.7797 & 0.7702 & 0.7825 & \textbf{0.7878} \\
 \hline
 \textbf{Pre-trained} & - & 0.5542 & 0.6344 & 0.4631 & \textbf{0.6697} & 0.6423 \\
\end{tabular}
\end{ctucolortab}
\caption{F1 scores of pre-trained models with and without further fine-tuning.}
\label{table:4}
}
\end{table}

\subsection{Pre-training + Evaluating}
To answer the question \hyperref[Q1]{Q1}, the pre-trained models from the previous section are evaluated on BABE without any fine-tuning on it. This way, it can be studied how well each model trained on each set can transfer knowledge to the detection of \gls{mb} in BABE. Thus, possibly unveils the relatedness to BABE. The results can be seen in table \ref{table:4}. In the table, this pre-training \textbf{without} further fine-tuning is referred to as \textbf{Pre-trained}.

Models pre-trained on Wikipedia data, both \textbf{WIKI} and \textbf{WNC} perform relatively well compared to \textbf{MB} and \textbf{SUBJ}. This suggests that the bias distribution in WIKI datasets is the closest to BABE.

The best performing model was pre-trained on the largest dataset, WNC. The model achieved an F1 score of 0.67 on the target dataset which is 21\% more than the model pre-trained on MB set

During pre-training, the F1 score of the WIKI and the MB set both peaked around 70\% on their validation sets; however, the model trained on MB generalized very poorly to BABE data as opposed to the WIKI model (0.46 against 0.63). I suspect that the low quality of the MB set and high size imbalance between the two sets played an important role in this result.

In conclusion, models that generalized to BABE best were pre-trained on Wikipedia datasets.

\subsection{Pre-training + Fine-tuning}
Secondly, pre-trained models are used as a sort of weight initialization for subsequent \textbf{fine-tuning} on the BABE. The results can be seen in table \ref{table:4}. This process is referred to as \textbf{Pre-trained + Fine-tuned}.

Fine-tuning a model pre-trained on \textbf{ALL} datasets combined resulted in the best performance; however, virtually the same performance was achieved by the model trained on the \textbf{SUBJe} combination. Importantly, the SUBJ split represents almost half of all data (see \ref{fig:cz_data}). Therefore, I assume that the performance of ALL model is high because of the presence of SUBJe in the training data.

Using the rest of the pre-trained models for fine-tuning actually hurt the performance. Yet, the difference is very small. The lowest score is achieved by fine-tuning the MB model. I suspect this is mainly due to the small size (2500 sentences) of the balanced MB set that may have led to overfitting.

In conclusion, using models pre-trained on the subjective datasets improved the performance of fine-tuning on BABE, but the increase was very small.




\section{Final evaluation}\label{classifier}
The final FERNET-C5 model has been pre-trained with \textbf{ALL} datasets combination and fine-tuned with the optimal parameters (\ref{hyperparams}) on BABE target dataset. Finally, on a \textbf{test} set it achieved an F1 score of \textbf{0.804}. A confusion matrix of predictions on the test set can be seen below \ref{fig:confmat}

\begin{figure}[h]
  \includegraphics[scale=0.5]{my_modules/multimedia/confmat.png}
  \caption{Confusion matrix of predictions on the test set.}
  \label{fig:confmat}
\end{figure}

For the final model that I share\footnote{\url{https://huggingface.co/horychtom/czech_media_bias_classifier}} on HuggingFace and which is used for inference experiments, the entire BABE dataset, including the test set, was used for training.

\section{Discussion}
The results suggest that subjectivity bias as opposed to media bias appears to be a bit more explicit and straightforward, since pre-training on the subjectivity task helped with \gls{mb} detection, but proved insufficient without further fine-tuning (only 55\% on BABE). This supports the assumption that \gls{mb} is composed of many more superficial linguistic features (\ref{features}).

Also, despite the relatively high performance, the final score on the test set is not representative due to its size. The test set consists of $\sim$ 500 sentences and therefore, may not adequately represent all bias information. For a better evaluation, I propose using a nested \gls{cv} \cite{stone1974cross}.

The authors of the original paper that introduces BABE \cite{spinde2021neural} report an F1 score of $\sim$ 0.8 for fine-tuned transformer model. That is approximately 2\% higher score than I achieved on the Czech version. This may be caused either by the noise introduced into the data during machine translation or by the possibly lower quality of the Czech pre-trained models.

Perhaps, a complete study with more models could be performed, but that would require an enormous number of trained models. Essentially, these results show that there was a minimal gain over the baseline (\textbf{+0.7\%}).

The results indicate that the overall low quality and limited size of the available datasets make their use for media bias detection impractical.








\chapter{Inference on Czech News sample}\label{inference}

\begin{itemize}
    \item abstract,headline, text bias correlation
    \item vbias progress over time for one domain
    \item plot the difference between domains
    \item plot the difference between topics
    \item info about 'commentary' articles is presented! super
    \item length correlates with bias -> perhaps longer text leaves more space for bias
    \item quoting has no correlation with bias
    \item čím víc článků je, tím menší je bias.. :(
\end{itemize}

vhodný články
\begin{enumerate}
    \item \url{https://www.novinky.cz/domaci/clanek/jak-cist-volby-294694}
    \item tahle je asi nej ta pod timhle
    \item \url{https://www.novinky.cz/zahranicni/evropa/150611-v-rakouskych-volbach-podle-pruzkumu-triumfuji-pravicovi-populiste.html}
\end{enumerate}

An example of classified news article can be seen in figure \ref{fig:demoarticle}.

\begin{figure}
\makebox[\textwidth][c]{
  \includegraphics[scale=0.5]{my_modules/multimedia/article_example.png}
  \caption{Example of classified article.}
  \label{fig:demoarticle}
  }
\end{figure}

\subsection{Demo}
Additionally, I provide a simple web demo for the reader to experiment with\footnote{\url{https://huggingface.co/spaces/horychtom/czech_media_bias_detection}}. The use of the demo can be seen in \ref{fig:demodemo}. The backend runs on HuggingFace's spaces\footnote{\url{https://huggingface.co/spaces}} which is a free hosting service for demonstration of \gls{ml} applications. 

The user can insert arbitrary text in Czech language; text is then split into sentences and classified individually. Then the average percentage bias score of the text is displayed. An \textit{interpret} button allows the user to further inspect the classification.

\begin{figure}
\makebox[\textwidth][c]{
  \includegraphics[scale=0.3]{my_modules/multimedia/bias.png}
  \caption{Example of the bias classifier demo usage. Red and blue color represent the biased and unbiased annotation respectively.}
  \label{fig:demodemo}
  }
\end{figure}


\chapter{Conclusion}
In this work I collected and analyzed all the literature and resource for studying state-of-the-art media bias detection and performed a minor experiment focused on gender bias detection as well. I presented new czech parallel dataset derived from Wikipedia and, in addition, 9 parallel translated czech datasets for tackling the media bias detection in \textbf{Czech language}, one of them large scale (360k sentences).

I trained and tuned the BERT-based czech language model and achieved an F1 score of 80.4 \% on a small test subset of a target dataset. I performed experiments on combining different datasets for pre-training the model, to push the performance on validation set. Pre-training on Subjectivity detection datasets performed the best, however, all tuning performed has had generally very low effect on performance +0.7\%. 

Finally, the final classifier has been used to build a publicly available demo and to analyze a sample of articles from "INSERT SOURCE" throughout the 2002 - 2018 period of time. Results of this study showed a trend in progression of media bias in time and suggested/revelaed a positive correlation between bias of the headline and the average bias of the article.



\section{Future perspective}
Even though \Gls{mtl} approach seems promising, it would require a lot of work on task selection and task evaluation, which is not focus of this thesis.analysis of decisions, creation of ground-truth dataset. 

As discussed in the experiments section, reasearch suggests that multitask learning increases classification accuracy significantly ref. Multitask model environment requires a lot of tasks \cite{aribandi2021ext5} to perform better than single task models. Therefore, for czech language setting, one of the future research possibilities would be to leverage multi-task learning for current classifier improvement. 





\printindex

\bibliographystyle{ieeetr}
\bibliography{my_modules/citation}
\printglossary
\appendix
\chapter{Gender classification results}\label{gender_results}

\begin{figure}[H]
\makebox[\textwidth][c]{
  \includegraphics[scale=0.52]{my_modules/multimedia/cfmats.png}
  \caption{Confusion matrices of gender classifier on test sets. On the left on large scale validation dataset. On the right on a gold-label small test set.}
  \label{fig:gender_cfmats}
  }
\end{figure}


\begin{figure}[H]
\makebox[\textwidth][c]{
  \includegraphics[scale=0.3]{my_modules/multimedia/gender_classifier.jpg}
  \caption{Example of the gender bias classification.}
  \label{fig:gender_classification}
  }
\end{figure}



\chapter{Training parameters}\label{all_parameters}

\begin{itemize}
  \item per\_device\_train\_batch\_size: 32
  \item gradient\_accumulation\_steps: 1
  \item learning\_rate: 3e-05
  \item weight\_decay: 0.1
  \item adam\_beta1: 0.9
  \item adam\_beta2: 0.999
  \item adam\_epsilon: 1e-08
  \item max\_grad\_norm: 1.0,
  \item num\_train\_epochs: 3,
  \item lr\_scheduler\_type": "linear"
  \item warmup\_ratio: 0.0
  \item seed: 42
  \item no\_cuda: false
  \item fp16: false
  \item remove\_unused\_columns: true
  \item load\_best\_model\_at\_end: false
  \item ignore\_data\_skip": false
  \item label\_smoothing\_factor: 0.0
  \item adafactor: false
\end{itemize}



\end{document}