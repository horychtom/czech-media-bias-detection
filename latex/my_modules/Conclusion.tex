\chapter{Conclusion}
In this work, I collected and analyzed the literature and resources for studying state-of-the-art media bias detection and also performed a minor experiment focused on gender bias detection. I presented a new Czech parallel dataset derived from Wikipedia and, in addition, nine parallel translated Czech datasets to tackle the detection of media bias in \textbf{Czech language}, one of them large-scale (360k sentences).

I trained and tuned the BERT-based FERNET-C5 language model and achieved an F1 score of 80.4 \% on a small test subset of a target dataset. I performed experiments on combining different datasets for pre-training the model to push the performance on the validation set. Pre-training on the subjectivity detection datasets performed the best; however, all the tuning performed generally had a very low effect on performance +0.7\%. 

Finally, the final classifier has been used to build a publicly available demo and to analyze a sample of articles from "INSERT SOURCE" throughout the 2002 - 2018 period of time. The results of this study showed a trend in the progression of media bias in time and revealed a positive correlation between the headline bias and the average bias of the article.

\section{Ethical Concerns}


\section{Future perspective}
As outlined in the introduction, media bias appears to be a combination of several potentially independent tasks, such as sentiment, agression, or subjective analysis. In this thesis, this hypothesis has also been somewhat supported by the result that pre-training on subjectivity task had the best influence on detection of \gls{mb} but a single-task model for subjectivity detection eventually performed worse than others. Therefore, as Spinde et al. \cite{spindeexploiting} suggest, the multi-task approach could be used to enhance the classification ability of the current Czech classifier.

Nevertheless, the current Czech classifier relies heavily on translated datasets, therefore, I suggest that for the future improvement, a construction of an original gold-standard Czech dataset is essential.
