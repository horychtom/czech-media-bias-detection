\chapter{Datasets} \label{datasets}
Due to the complex nature of \gls{mb}, different datasets try to capture different aspects of it. In this section, I present a collection of \textbf{all} datasets related to biased writing and subjectivity detection available. Later this collection is used to study the proposed research questions \hyperref[Q1]{Q1} and \hyperref[Q2]{Q2} and to build a final classifier. For details see experiment section \ref{experiments}.

As stated above, this work focuses only on sentence-level classification; thus, data annotated on article-level were not considered.

I divided the available datasets into 3 main families: 
\begin{itemize}
    \item Subjectivity bias
    \item Wikipedia bias
    \item Media bias
\end{itemize}
Wikipedia bias could also be considered as a form of subjective bias, but all the Wiki data come from the same distribution\footnote{some are even different samples from the same larger corpora}. and environment; hence I find it reasonable to put them together.


\section{Subjectivity Datasets}

\subsection{SUBJ}
It is reasonable to include datasets that focus on the detection of subjectivity since it is one of the \gls{mb} characteristics. The Subjectivity dataset (SUBJ) \cite{Pang+Lee:04a} consists of 10000 sentences gathered from movie review sites. Sentences are labeled as subjective and objective with 1:1 ratio. 

The data were collected in an automatic way. The authors made an assumption that all reviews from Rottentomatoes\footnote{https://www.rottentomatoes.com/} are subjective, and all plot summaries from IMBD\footnote{ www.imdb.com} are objective. For each class, 5k sentences were sampled \textbf{randomly}.




\subsection{MPQA}
\textbf{M}ulti-\textbf{P}erspective \textbf{Q}uestion \textbf{A}nswering (MPQA) Opinion corpus is another dataset that can be used for subjectivity detection. I used the MPQA Opinion corpus version 2.0, which consists of 692 articles from 187 different news sources. In total 15802 sentences. All articles are from June 2001 to May 2002.

The corpus offers a rich annotation scheme \cite{wiebe2005annotating} that focuses on sentiment and subjectivity annotations.

To extract the bias information, I focused on two types of annotations:
\begin{itemize}
    \item Direct subjective
    \item Expressive subjective
\end{itemize}
These annotations were present if any form of subjectivity was suspected by the annotator. Each annotation consists of indices of span in the text and properties. For each sentence in the corpus, I extracted labels as follows:

If there was at least one annotation \textbf{direct\_subjective} or \textbf{expressive\_subjectivity} with span inside the sentence and the intensity tag was not $low$, the sentence was labelled as \textit{subjective $\sim$ biased}. All other sentences were extracted as \textit{objective $\sim$ unbiased}.

This approach has produced $9484$ subjective sentences and 6318 objective sentences.


\section{Media Bias datasets}


\subsection{BASIL}
BASIL dataset \cite{fan2019plain} comprises 300 articles with 1727 sentence level bias annotations. The authors of the dataset distinguish between \textbf{lexical} and \textbf{informational} bias.

The annotations were performed by two experts and further resolution discussions have later led to 0.56 and 0.7 \Gls{iaa} score for lexical and informational bias, respectively.

Even though BASIL brings sufficient annotation quality, most of the labeling resulted in informational bias annotations, leaving only 478 sentences for the lexical bias class. Informational bias requires a different approach to detection \cite{van2020context} and usually depends dramatically on the context. Therefore, I extracted all sentences with the informational label as a neutral class.




\subsection{Ukraine Crisis Dataset}
This dataset \cite{farber2020multidimensional} offers 2057 sentences with binary media bias labels. All sentences are related to one topic - the Ukraine-Russian crisis. Data were gathered from 90 news sources.

The authors introduce rich annotations for each sentence. Each of them looks at the bias from a different perspective, called \textit{bias dimensions}:
\begin{enumerate}
    \item Hidden Assumptions and Premises
    \item Subjectivity
    \item Framing
\end{enumerate}
In addition, the \textit{overall bias} annotation is presented. Together, the data include 44547 fine-grained annotations. For simplicity, I only included the overall bias annotation.
Even though this dataset encompasses comprehensive bias information, it also suffers from a low \Gls{iaa} score. Specifically, Krippendorff’s $\alpha = -0.05$.



\subsection{NFNJ}
The NFNJ\footnote{\cite{farber2020multidimensional} refers to this dataset as NFNJ, however, in the original paper the name is not presented.} dataset provides 966 sentences from 46 articles with annotations on a fine-grained level.

Authors share the dataset for research purposes; however, the public version differs from the one described in the original paper. Therefore, while extracting the final dataset, I made a few assumptions:

In the raw data, contributions from multiple annotators on each sentence are provided. Therefore, I extracted the labels as a simple arithmetical mean of the labels. Furthermore, the original labels stand for 
\begin{itemize}
    \item 1: 'neutral'
    \item 2: 'slightly biased but acceptable'
    \item 3: 'biased'
    \item 4: 'very biased'
\end{itemize}
To obtain the final labels in an unbiased-biased format, I simply assumed sentences with mean-score $\leq$ 2 as neutral and $>$ 2 as biased.

The Fleiss Kappa \Gls{iaa} score averaged at zero, making it practically unusable as a standalone dataset.



\subsection{BABE}
\textbf{B}ias \textbf{A}nnotations \textbf{B}y \textbf{E}xperts (BABE)\cite{spinde2021neural} is a key media bias dataset from \Gls{mbg}\footnote{\url{https://media-bias-research.org/}}, which is to the best of my knowledge, the highest quality media bias dataset to this day. It builds on top of MBIC \cite{Spinde2021MBIC} which is a smaller crowd-sourced dataset.

BABE contains 3700 sentences. 1700 sentences are from MBIC, which were extracted from 1000 news articles, and in addition extended by 2000 more sentences, altogether covering 12 topics, annotated with binary bias indications. In addition, the annotations were enriched with a list of biased words. However, the presence of biased words does not always result in an overall biased sentence label. See \ref{table:2} for examples.

It has been annotated by eight experts resulting in \gls{iaa} Krippendorfs $\alpha = 0.39$, which exceeds other media bias datasets by a significant margin. It also provides detailed information about the annotator background, making it a \textbf{reliable} source of information. The pipeline of the collection of BABE can be seen in \ref{fig:babe-data}.

This dataset plays a pivotal role in my approach to media bias detection and is selected as a target for tuning and evaluating language models in chapter \ref{experiments}. Examples of BABE data points can be seen in 
\ref{table:2}

\begin{table}
\makebox[\textwidth][c]{
\begin{tabular}[\textwidth]{l|c}
\hline
\textbf{sentence} & \textbf{label} \\
\hline
\cellcolor{biased_clr} Americans \colorbox{biasedw_clr}{know} President Donald Trump is an \colorbox{biasedw_clr}{outrageous}, \colorbox{biasedw_clr}{scandal-ridden} character. & \cellcolor{biased_clr} biased \\
\hline
\cellcolor{unbiased_clr} Biden said he would seek Muslims to serve in his administration. & \cellcolor{unbiased_clr} unbiased \\
\hline
\cellcolor{biased_clr} Biden’s shift \colorbox{biasedw_clr}{radically} leftward reflects that of his party. & \cellcolor{biased_clr} biased \\
\hline
\cellcolor{unbiased_clr} Anti-vaccine groups take \colorbox{biasedw_clr}{dangerous} online \colorbox{biasedw_clr}{harassment} into the real world. & \cellcolor{unbiased_clr} unbiased \\
\hline
\end{tabular}
}
\caption{Example of biased and unbiased sentences from \textbf{BABE}}
\label{table:2}
\end{table}


\begin{figure}
  \includegraphics[scale=0.3]{my_modules/multimedia/babe_workflow.png}
  \caption{Data collection and annotation pipeline of \textbf{BABE}, reprinted from \cite{Spinde2021f}}
  \label{fig:babe-data}
\end{figure}



\section{Wikipedia datasets}\label{wiki-npov}
Due to annotation costs and the overall lack of large-scale datasets in the media bias setting, many researches \cite{pryzant2020automatically,recasens2013linguistic,hube2019neural} used Wikipedia's \Gls{npov} policy\footnote{\url{https://en.wikipedia.org/wiki/Wikipedia:Neutral_point_of_view}} to construct a large-scale corpora \textbf{automatically}. 

Wikipedia's NPOV policy is a set of rules that aim to preserve neutrality in Wikipedia articles. Some examples of NPOV principles are as follows:
\begin{itemize}
    \item Avoid stating opinions as facts.
    \item Avoid stating facts as opinions.
    \item Prefer nonjudgmental language.
\end{itemize}

When neutrality is contested, a Wikipedia article can be moved to NPOV dispute by tagging it with \{\{NPOV\}\} or \{\{POV\}\}\footnote{Other POV related variations are often used.} template. Debate on specific details of neutrality violations is then initialized among editors and eventually resolved, leading to the removal of the tag.

This editorial information can be leveraged to extract parts of the text that violate the NPOV and their unbiased counterparts. However, it has been shown \cite{hube2019neural,zhong-etal-2021-wikibias-detecting} that such automatic extraction can suffer from noisy labeling. In some cases \cite{hube2019neural} up to 60\% of data points from the positive class were actually neutral.

Even though these datasets introduce a large number of samples that are highly related to media bias, they are all sampled from Wikipedia's environment, which can be very different from the news environment. For this reason, a dataset based \textbf{only} on Wikipedia should not be used for training a final classifier of the \textbf{news text}.




\subsection{Wiki Neutrality Corpus}\label{wiki}
\Gls{wnc} \cite{pryzant2020automatically} is a parallel corpus of 180k pairs of biased and unbiased sentences. For the collection of the data, \ref{wiki-npov} approach was adopted. The authors crawled revisions from 2014 to 2019. Each revision has been processed to check if it contains any variation of \textit{POV} related tag inside. This approach yielded 180k pairs such that the sentence before edit is considered biased and the modified/added sentence after edit is considered neutral/unbiased.
    
In addition to WNC, 385k of sentences that have not been changed during the NPOV dispute were extracted as neutral and for word-level classification purposes, a subset of the WNC corpus, where only one word is changed in the biased-unbiased pair, were added.




\subsection{CW-HARD}
Hube et al. \cite{hube2019neural} constructed a dataset based on NPOV, where only revisions with one sentence diff were filtered. However, because of the potentially noisy outcome, 5000 sentences were sampled and annotated using crowdsourcing. However, the Krippendorff's Alpha agreement score measured only $\alpha = 0.124$, which is generally considered low. 

After filtering out sentences that annotators labeled with the "I don't know" option, the final dataset consists of 1843 statements labeled as biased and 3109 labeled as neutral, a total of 4953 sentences.




\subsection{WikiBias}
This is the latest dataset based on Wikipedia. The authors \cite{zhong-etal-2021-wikibias-detecting} closely follow the approach of WNC \cite{pryzant2020automatically} and extract another parallel wiki corpus of 214k sentences.
To achieve a higher quality corpus, 4099 sentence pairs were randomly sampled and labeled by trained annotators. As a result, introduced \textbf{WikiBias-Manual} dataset consists of 3400 biased and 4798 neutral sentences annotated with high \gls{iaa} score of Cohen's $\kappa = 0.734$

\begin{table}
\begin{ctucolortab}
\begin{tabular}{c|c|c|c}
 \textbf{Dataset} & \textbf{Size} & \textbf{Annotation} & \textbf{Agreement}\\
 \hline
 \textbf{SUBJ} & 10.000 & automatic & -\\ 
 \hline
 \textbf{MPQA} & 15.802 & annotators & high \\
 \hline
 \textbf{BASIL} &  1.727 & annotators & medium \\ 
 \hline
 \textbf{Ukraine Crisis Dataset} & 2.057 & crowdsourcing & low \\ 
 \hline
 \textbf{NFNJ} & 888 & crowdsourcing & low \\
 \hline
 \textbf{BABE} & 3673 & annotators & medium \\
 \hline 
 \textbf{WNC} & 362.990 & automatic & - \\
 \hline
 \textbf{CW-hard} & 4953 & crowdsourcing & low \\
 \hline 
 \textbf{WikiBias} & 8198 & annotators & high \\
 \hline
\end{tabular}
\end{ctucolortab}
\caption{Comparison of all bias related datasets collected}
\label{table:1}
\end{table}



\section{Unused datasets}
 Some datasets focus on a slightly different task yet still carry potentially useful information. Such data can be useful in a \gls{mtl} setting \ref{mtl}. To name a few which are focused on the detection of ideology:
\begin{itemize}
\item \textbf{NewsB} - 
Consists of labels capturing the authors political ideology (liberal, conservative) Labeled through distant supervision.
\item \textbf{IBC} - Also focuses on ideology detection; however, it is not publicly available.
\end{itemize}


\section{Summary}
In the previous section, I introduced all resources that are potentially useful for media bias analysis and are publicly available. The overview of all datasets and its properties can be seen in figure \ref{table:1}.

BABE dataset is generally a good benchmark, and its translated parallel version will be used for evaluation in this work. Training a classifier on combinations of different datasets is studied in \ref{experiments}. Unfortunately, a lot of the data suffer from noisy labeling and low \gls{iaa} greatly. Therefore, its usability is considerably limited.
